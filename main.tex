\documentclass[journal,12pt,twocolumn]{IEEEtran}

\usepackage{setspace}
\usepackage{gensymb}

\singlespacing 



\usepackage[cmex10]{amsmath}

\usepackage{amsthm}

\usepackage{mathrsfs}
\usepackage{txfonts}
\usepackage{stfloats}
\usepackage{bm}
\usepackage{cite}
\usepackage{cases}
\usepackage{subfig}

\usepackage{longtable}
\usepackage{multirow}

\usepackage{enumitem}
\usepackage{mathtools}
\usepackage{steinmetz}
\usepackage{tikz}
\usepackage{circuitikz}
\usepackage{verbatim}
\usepackage{tfrupee}
\usepackage[breaklinks=true]{hyperref}
\usepackage{graphicx}
\usepackage{tkz-euclide}
\usepackage{float}

\usetikzlibrary{calc,math}
\usepackage{listings}
    \usepackage{color} %%
    \usepackage{array} %%
    \usepackage{longtable} %%
    \usepackage{calc} %%
    \usepackage{multirow} %%
    \usepackage{hhline} %%
    \usepackage{ifthen} %%
    \usepackage{lscape}     
\usepackage{multicol}
\usepackage{chngcntr}
\newcommand{\norm}[1]{\left\lVert#1\right\rVert}

\DeclareMathOperator*{\Res}{Res}

\renewcommand\thesection{\arabic{section}}
\renewcommand\thesubsection{\thesection.\arabic{subsection}}
\renewcommand\thesubsubsection{\thesubsection.\arabic{subsubsection}}

\renewcommand\thesectiondis{\arabic{section}}
\renewcommand\thesubsectiondis{\thesectiondis.\arabic{subsection}}
\renewcommand\thesubsubsectiondis{\thesubsectiondis.\arabic{subsubsection}}


\hyphenation{op-tical net-works semi-conduc-tor}
\def\inputGnumericTable{} %%

\lstset{
%language=C,
frame=single, 
breaklines=true,
columns=fullflexible
}
\begin{document}


\newtheorem{theorem}{Theorem}[section]
\newtheorem{problem}{Problem}
\newtheorem{proposition}{Proposition}[section]
\newtheorem{lemma}{Lemma}[section]
\newtheorem{corollary}[theorem]{Corollary}
\newtheorem{example}{Example}[section]
\newtheorem{definition}[problem]{Definition}

\newcommand{\BEQA}{\begin{eqnarray}}
\newcommand{\EEQA}{\end{eqnarray}}
\newcommand{\define}{\stackrel{\triangle}{=}}
\newcommand\hlight[1]{\tikz[overlay, remember picture,baseline=-\the\dimexpr\fontdimen22\textfont2\relax]\node[rectangle,fill=blue!50,rounded corners,fill opacity = 0.2,draw,thick,text opacity =1] {$#1$};}
\bibliographystyle{IEEEtran}
\providecommand{\mbf}{\mathbf}
\providecommand{\pr}[1]{\ensuremath{\Pr\left(#1\right)}}
\providecommand{\qfunc}[1]{\ensuremath{Q\left(#1\right)}}
\providecommand{\sbrak}[1]{\ensuremath{{}\left[#1\right]}}
\providecommand{\lsbrak}[1]{\ensuremath{{}\left[#1\right.}}
\providecommand{\rsbrak}[1]{\ensuremath{{}\left.#1\right]}}
\providecommand{\brak}[1]{\ensuremath{\left(#1\right)}}
\providecommand{\lbrak}[1]{\ensuremath{\left(#1\right.}}
\providecommand{\rbrak}[1]{\ensuremath{\left.#1\right)}}
\providecommand{\cbrak}[1]{\ensuremath{\left\{#1\right\}}}
\providecommand{\lcbrak}[1]{\ensuremath{\left\{#1\right.}}
\providecommand{\rcbrak}[1]{\ensuremath{\left.#1\right\}}}
\theoremstyle{remark}
\newtheorem{rem}{Remark}
\newcommand{\sgn}{\mathop{\mathrm{sgn}}}
%\providecommand{\abs}[1]{\left\vert#1\right\vert}
\providecommand{\res}[1]{\Res\displaylimits_{#1}} 
\providecommand{\norm}[1]{$\left\lVert#1\right\rVert$}
%\providecommand{\norm}[1]{\lVert#1\rVert}
\providecommand{\mtx}[1]{\mathbf{#1}}
%\providecommand{\mean}[1]{E\left[ #1 \right]}
\providecommand{\fourier}{\overset{\mathcal{F}}{ \rightleftharpoons}}
%\providecommand{\hilbert}{\overset{\mathcal{H}}{ \rightleftharpoons}}
\providecommand{\system}{\overset{\mathcal{H}}{ \longleftrightarrow}}
 %\newcommand{\solution}[2]{\textbf{Solution:}{#1}}
\newcommand{\solution}{\noindent \textbf{Solution: }}
\newcommand{\cosec}{\,\text{cosec}\,}
\providecommand{\dec}[2]{\ensuremath{\overset{#1}{\underset{#2}{\gtrless}}}}
\newcommand{\myvec}[1]{\ensuremath{\begin{pmatrix}#1\end{pmatrix}}}
\newcommand{\mydet}[1]{\ensuremath{\begin{vmatrix}#1\end{vmatrix}}}
\numberwithin{equation}{subsection}
\makeatletter
\@addtoreset{figure}{problem}
\makeatother
\let\StandardTheFigure\thefigure
\let\vec\mathbf
\renewcommand{\thefigure}{\theproblem}
\def\putbox#1#2#3{\makebox[0in][l]{\makebox[#1][l]{}\raisebox{\baselineskip}[0in][0in]{\raisebox{#2}[0in][0in]{#3}}}}
     \def\rightbox#1{\makebox[0in][r]{#1}}
     \def\centbox#1{\makebox[0in]{#1}}
     \def\topbox#1{\raisebox{-\baselineskip}[0in][0in]{#1}}
     \def\midbox#1{\raisebox{-0.5\baselineskip}[0in][0in]{#1}}
\vspace{3cm}
\title{Assignment No.6}
\author{Panisha Gundelli}
\maketitle
\newpage
\bigskip
\renewcommand{\thefigure}{\theenumi}
\renewcommand{\thetable}{\theenumi}
Download latex-tikz codes from
\begin{lstlisting}
https://github.com/Panisha707/ASSIGNMENT07/blob/main/main.tex
\end{lstlisting}
%
Question taken from
\begin{lstlisting}
Optimization , exercises 2.22
\end{lstlisting}
\section{Question No 1}
Find the shortest distance of the point \myvec{0\\c} from
the parabola $y = x^2$, where $\frac{1}{2}$ \leq c \leq 5
\section{Solution}
Let the points $\vec{P} =\myvec{0\\c}$ and $\vec{Q} =\myvec{x&y}$ on parabola $y=x^2$
\\Now the distance between the points can be calculated by
\begin{align}
  \vec{P}\vec{Q}=\sqrt{\myvec{x-0}^2+\myvec{y-c}^2} 
\end{align}
As the equation of parabola is $y=x^2$, we can rewrite the above equation as
 \begin{align}
      \vec{P}\vec{Q}=\sqrt{\myvec{x-0}^2+\myvec{x^2-c}^2}\\
      \vec{P}\vec{Q}^2=\myvec{x}^2+\myvec{x^2-c}^2
 \end{align}
Let $f(x) = x^2+\myvec{x^2-c}^2$ 
\begin{align}
    f(x) = x^2+\myvec{x^2-c}^2\\
    f^\prime(x)=2x+2\myvec{x^2-c}2x\\
    =2x\myvec{1+2x^2-2c}\\
     =4x\myvec{x^2-c+\frac{1}{2}}\\
     =4x\myvec{x-\sqrt{c-\frac{1}{2}}}\myvec{x+\sqrt{c-\frac{1}{2}}}
\end{align}
when $c>\frac{1}{2}$,
For maxima, put $f^\prime(x)=0$
\begin{align}
   4x\myvec{x^2-c+\frac{1}{2}}=0 \\
   \implies x=0, x=\pm \sqrt{c-\frac{1}{2}}
\end{align}
Now,
\begin{align}
   f^{\prime\prime}(x)=4\myvec{x^2-c+\frac{1}{2}}+4x\myvec{2x}
\end{align}
At,
\begin{align}
    x =\pm \sqrt{c-\frac{1}{2}} , f^{\prime\prime}\geq 0
\end{align}
$\therefore f^{\prime\prime}(x)$ is minimum\\
Hence, minimum value of $f(x)=\norm{\vec{P}\vec{Q}}$
\begin{align}
   = \sqrt{\myvec{\sqrt{c-\frac{1}{2}}}^2+\myvec{\myvec{\sqrt{c-\frac{1}{2}}}-c}^2}\\
   = \sqrt{c-\frac{1}{2}+\myvec{c-\frac{1}{2}-c}^2}\\
    =\sqrt{c-\frac{1}{4}} , \because \frac{1}{2}\leq c \leq 5 
    \end{align}





\end{document}

